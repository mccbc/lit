\documentclass[]{article}


\begin{document}
\thispagestyle{empty}

My work aligns with NASA's Exoplanet Exploration program, since it shares a common goal of characterizing planetary systems and earth-like planets. Exoplanet research is one of the fastest growing fields in astronomy---every increase in the sophistication of our observing capabilities necessitates a proportional increase in the complexity of the resultant data analysis. Spectral analysis of transiting exoplanets reveals information about the contents of their atmospheres, yet the dynamics of these environments and reasons for certain atomic abundances are not well understood. Radiation forces, for example, play a huge role in determining an atmosphere's composition and temperature profile; these forces can even exceed the gravitational forces on atoms in the upper atmosphere, causing an outflow that interacts with stellar winds. Each of these phenomena are intricately linked to the spectral signatures seen in observations by Hubble, Kepler, TESS, and GAIA, among others. Yet, by their very nature, these observations are limited to spectral analysis---there is no method by which one can understand hydrodynamics and radiative transport within these atmospheres except by comparison with numerical models. This is where my work enters the picture: the model of resonance-line radiation in optically thick media I put forward enables an accurate calculation of recombination, electron impact excitation, and absorption and emission by the gas. This naturally leads to a detailed 3-dimensional calculation of the transmission spectra, which can be directly compared to observations by the aforementioned telescopes.he resultant data analysis. Spectral analysis of transiting exoplanets reveals information about the contents of their atmospheres, yet the dynamics of these environments and reasons for certain atomic abundances are not well understood. Radiation forces, for example, play a huge role in determining an atmosphere's composition and temperature profile; these forces can even exceed the gravitational forces on atoms in the upper atmosphere, causing an outflow that interacts with stellar winds. Each of these phenomena are intricately linked to the spectral signatures seen in observations by Hubble, Kepler, TESS, and GAIA, among others. Yet, by their very nature, these observations are limited to spectral analysis---there is no method by which one can understand hydrodynamics and radiative transport within these atmospheres except by comparison with numerical models. This is where my work enters the picture: the model of resonance-line radiation in optically thick media I put forward enables an accurate calculation of recombination, electron impact excitation, and absorption and emission by the gas. This naturally leads to a detailed 3-dimensional calculation of the transmission spectra, which can be directly compared to observations by the aforementioned telescopes.

The model atmospheres and radiation intensities produced in this work will be used to create transmission spectra which can be directly compared with ground and space-based data for specific exoplanets, and will aid in making predictions for future observations. Previous work has established a baseline for atomic data, opacities, and stellar ultraviolet spectra expected from these types of transits, but sophisticated and quantitative analysis is still forthcoming. For example, recent data from the Space Telescope Imaging Spectrograph aboard the Hubble Space Telescope indicates that 5 percent of Lyman Alpha absorption from the planet HD209458B remains unaccounted for by even the maximal filling of the gravitational radius at which the planet can hold its atmosphere. Thus, material is escaping from the surface in a radiation-pressure-driven outflow---exactly the phenomenon my work simulates. My numerical simulations will form a unique link to advanced and complicated radiative processes that have yet to be treated in a complex manner by theorists in the field of exoplanetary research. The results of this novel model for radiative boundaries and Monte Carlo acceleration are necessary and timely to analyze past data and to inform future exoplanet observations.
\end{document}

