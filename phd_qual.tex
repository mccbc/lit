\documentclass[onecolumn]{aastex63}
\usepackage{amsmath}
\usepackage{listings}

\lstset{frame=tb,
  language=Python,
  showstringspaces=false,
  columns=flexible,
  basicstyle={\small\ttfamily},
  commentstyle={\small\ttfamily},
  breaklines=true,
  breakatwhitespace=true,
  tabsize=4
}

\usepackage[T1]{fontenc}
\newcommand{\vdag}{(v)^\dagger}
\newcommand\aastex{AAS\TeX}
\newcommand\latex{La\TeX}
\shortauthors{McClellan}
\graphicspath{{./}{figures/}}
\begin{document}

\title{PhD Qualifying Examination \\ \small{\normalfont Literature Review and Research Overview}}
\author{B. Connor McClellan}
\affiliation{University of Virginia}
\keywords{}

\tableofcontents

\section{Literature Review}

\begin{itemize}
    \item Convey the central points of the literature.
    \item Summarize the evidence and arguments for each point.
    \item Acknowledge the limitations and implications of the prior work reported in the literature.
    \item Develop suggestions for ways to resolve unanswered questions and follow up implications of the work.
\end{itemize}

\subsection{Definitions}
%% What are things I learned in order to understand this paper?
\begin{itemize}
    \item \textbf{Resonance line radiation} refers to spectral lines caused by an electron jumping between the first excited state and the ground state of an atom or ion. The Lyman Alpha (Ly$\alpha$) resonance line is the $n=2\rightarrow1$ transition of hydrogen, and is one of the most common mechanisms by which astrophysical gas can cool \citep{neufeld1990}. Stellar Ly$\alpha$ radiation is frequently scattered by H atoms in the ground state within exoplanetary atmospheres.
    \item The \textbf{Poisson equation} is an equation of the form
    \begin{equation}
        \nabla^2 \phi = f
    \end{equation}
    It is a second-order differential equation which frequently comes up in electrostatics and fluid dynamics.
    \item An \textbf{eigenfunction expansion} is a method of solving differential equations, which utilizes a set of orthogonal eigenfunctions to construct solutions to said equations. As more eigenfunctions are included in the set, the solution converges closer to the true solution of the equation.
    \item A \textbf{line profile} is the shape of a spectral line on an intensity v. frequency plot. In our work we use a Voigt profile, which is the convolution of a Lorentzian (line wing) and a Gaussian (Doppler core) profile. For our purposes, we ignore the Doppler core contribution and use only the Lorentzian, since we care most about the solutions' behavior in the line wing.
\end{itemize}



\subsection{Scientific Purpose}
%% What is the science question addressed by the paper?
It is shown that resonance-line radiation transfer in low-density, optically-thick media can be described by the Poisson equation. The mean number of scatterings for escape are calculated, as well as the center and surface line profiles.

\subsection{Methods}
%% What is the method used to address the science question?
The condition for this analytic solution is that the optical depth is so large, the transfer problem is dominated by the redistribution of radiation in the damping wings of the line. Modeling redistribution as a diffusion in frequency space naturally leads to a partial differential equation which turns out to be the Poisson equation. The resulting analytic solution is valid in the limit of large optical depth and matches well to numerical results.

%\subsection{Citeable Results}
%%% A list of citeable results in the paper
%\begin{itemize}
%    \item
%\end{itemize}
\subsection{Derivations}
%%% Derivations of any critical equations that appear in the paper

\begin{equation} \label{harrington1}
    \frac{1}{3\phi^2(x)}\frac{d^2J(\tau, x)}{d\tau^2} = J(\tau, x) - (1-\epsilon)\int_{-\infty}^{\infty}J(\tau, x')q(x, x')dx' - \frac{G(\tau)}{4\pi}
\end{equation}

The general redistribution function, $q(x, x') = R_{II-A}(x, x')/\phi (x)$, is expanded according to \cite{1971JQSRT..11.1365A}. Notably, they define the integrated complementary error function as 

\begin{equation}
    \mathrm{ierfc}(z) \equiv \int_z^{\infty} \mathrm{erfc}(t) dt = \frac{1}{\sqrt{\pi}} e^{-z^2} - z\  \mathrm{erfc}(z)
\end{equation}

which Harrington incorrectly writes down as ``$i\ \mathrm{erfc(...)}$''. The first terms in the expansion are 


\begin{equation}
    R_{II-A}(x, x') \approx \frac{1}{2}\mathrm{erfc}(|r|+|s|) + \left(1 - \frac{a^2}{s^2} + \frac{a^4}{s^4}\right) \frac{a}{\pi s^2} \mathrm{ierfc}(|r|) + ...
\end{equation}

with $r=(x-x')/2$ and $s=(x+x')/2$. The first term is ignored in the line wings (TODO: why?), and the remaining leading term of order $1/s^2$ is simply

\begin{equation}
    R_{II-A}(x, x') \approx \frac{a}{\pi s^2}\mathrm{ierfc}(|r|)
\end{equation}

Division by the line profile, approximated as $\phi(x) \approx a/(\pi x^2)$, yields

\begin{equation}
    q(x, x') = \frac{x^2}{s^2} \mathrm{ierfc}(|r|) \approx \left(1+\frac{2r}{x}\right)\left(\frac{1}{\sqrt{\pi}}e^{-r^2} - |r|\ \mathrm{erfc}(|r|)\right)
\end{equation}

using the binomial approximation to obtain $x^2/s^2 = (r+s)^2/s^2 = (1+r/s)^2 \approx 1 + 2r/s \approx 1 + 2r/x$. The last step in this approximation seems to come from the understanding that as long as $|x-x'|\ll 1$, $s \approx x$. However, if this approximation was used in the original expression $x^2/s^2$, the coefficient of the integrated error function would simply be 1.

The $J(x')$ term in the integral in Eq. \ref{harrington1} is then expanded in a Taylor series about $x$.

\begin{equation}
    J(x') \approx J(x) + \frac{dJ(x)}{dx}(x' - x) + \frac{1}{2}\frac{d^2J(x)}{dx^2}(x'-x)^2 + ...
\end{equation}

Changing variables to $r$, the integral becomes

\begin{equation}
    2 \int_{-\infty}^{\infty} \left(1+\frac{2r}{x}\right)\left(\frac{1}{\sqrt{\pi}}e^{-r^2} - |r|\ \mathrm{erfc}(|r|)\right)\left[J(x) - 2\frac{dJ(x)}{dx}r + 2\frac{d^2J(x)}{dx^2}r^2 + ...\right] dr
\end{equation}

where the minus sign introduced by the change of variables has been used to flip the bounds of integration, which would nominally go from $\infty$ to $-\infty$ under the variable $r$.

The various combinations of terms evaluate as the following:

\begin{itemize}
    \item $\int_{-\infty}^{\infty} e^{-r^2} dr = \sqrt{\pi}$
    \item $\int_{-\infty}^{\infty} r^{n} e^{-r^2} dr = 0$ for all odd powers $n$
    \item $\int_{-\infty}^{\infty} r^2 e^{-r^2} dr = \sqrt{\pi}/2$
    \item $\int_{-\infty}^{\infty} |r|\ \mathrm{erfc}(|r|)\ dr = 1/2$
    \item $\int_{-\infty}^{\infty} r^n |r|\ \mathrm{erfc}(|r|)\ dr = 0$ for all odd powers $n$
    \item $\int_{-\infty}^{\infty} r^2 |r|\ \mathrm{erfc}(|r|)\ dr = 3/8$
\end{itemize}

Therefore, collecting nonzero terms and evaluating the integral, we are left with

\begin{equation}
    \begin{split}
    \int_{-\infty}^{\infty}J(x')q(x, x') &= 2 \int_{-\infty}^{\infty}
    \frac{1}{\sqrt{\pi}}e^{-r^2}J(x) 
    + \frac{2}{\sqrt{\pi}}r^2e^{-r^2} \frac{d^2J(x)}{dx^2}
    - \frac{4}{x\sqrt{\pi}}r^2e^{-r^2} \frac{dJ(x)}{dx}\\
    &\hspace{1.5cm}- |r|\ \mathrm{erfc}(|r|)\ J(x)
    - 2r^2\ |r|\ \mathrm{erfc}(|r|)\ \frac{d^2J(x)}{dx^2} \\
    &\hspace{1.5cm}+ \frac{4}{x}r^2\ |r|\ \mathrm{erfc}(|r|)\ \frac{dJ(x)}{dx}
    dr\\ \\
    & = 2\left[J(x) + \frac{d^2J(x)}{dx^2} - \frac{2}{x}  \frac{dJ(x)}{dx} -\frac{1}{2}J(x) - \frac{3}{4} \frac{d^2J(x)}{dx^2} + \frac{3}{2x}\frac{dJ(x)}{dx}\right]\\
    & = J(x) - \frac{1}{x}\frac{dJ(x)}{dx} + \frac{1}{2}\frac{d^2J(x)}{dx^2}
   \end{split}
\end{equation}

\subsection{Discussion and Concerns}
%% Are there any red flags?

%\subsection{Personal Relevance}
%%% Place the work into a familiar context. How is the discussion in the paper related to your own research interests and experience?

%\subsection{Summary}
%%% Reread the abstract, and offer summarizing commentary.
% --------------------------- %

\section{Research Overview}

\begin{itemize}
    \item Demonstrate an appropriate knowledge of the field in the area of research.
    \item Demonstrate a clear understanding of the research goals.
    \item Show a reasonable approach to the research goal.
    \item Show creativity in the approach to the research.
    \item Outline substantial progress toward the research goal during the year.
\end{itemize}


In this section I will outline the numerical work I have done since Summer 2020, and provide the analytic derivations that motivated some of that work.

\subsection{Background}

\subsection{Motivation}

\subsection{Methods}

\subsection{Results to Date}

\subsection{Two-Stream Approximation Boundary Condition}

In the two-stream approximation for radiative transfer, it is assumed that the intensity is nearly isotropic---that is, it can be broken up into two components, $I_+(\tau)$ and $I_-(\tau)$. These will be referred to as the ``up'' and ``down'' intensities, respectively, and written shorthand as $I_+$ and $I_-$. The intensity is a function of the optical depth $\tau$ and angle $\mu = \cos{\theta}$, and is expressed in up and down components as

\begin{equation}
    I(\tau, \mu) = I_+ \delta(\mu - 1/\sqrt{3}) + I_- \delta(\mu + 1/\sqrt{3})
\end{equation}

The delta functions in angle are offset by $\pm 1/\sqrt{3}$, due to the normalization condition that the radiation pressure must be equal to $1/3$ the photon energy density. Under addition in quadrature, this condition is satisfied. As will be explored in the review of \cite{2006ApJ...649...14D}'s closed-form solution, a less rigorous approach is to offset the angular delta functions by $1/2$, which yields a different surface boundary condition.  The mean intensity is calculated as follows.

\begin{equation}
    \begin{split}
    J(\tau) &= \frac{1}{4\pi}\int d\Omega\ I(\tau, \mu)\\ 
    &= \frac{1}{2}\int_{-1}^1 d\mu\ I(\tau, \mu)\\
    &= \frac{1}{2}\int_{-1}^1 d\mu\ \left[I_+ \delta(\mu - 1/\sqrt{3}) + I_- \delta(\mu + 1/\sqrt{3})\right]\\
    &= \frac{I_+ + I_-}{2}
    \end{split}
\end{equation}

This makes intuitive sense; the mean intensity is the average of the ingoing and outgoing intensity. To calculate the flux, the intensity is integrated with an additional angular moment.


\begin{equation}
    \begin{split}
        H(\mu) &= \frac{1}{4\pi}\int d\Omega\ \mu\ I(\tau, \mu)\\
        &= \frac{1}{4\pi}\int_0^{2\pi} d\phi \int_0^1 d\mu\ \mu\ I(\tau, \mu)\\
        &= \frac{1}{2} \int_{-1}^1 d\mu\ \mu\ I(\tau, \mu)\\
        &= \frac{1}{2} \int_{-1}^1 d\mu\ \left[\mu I_+ \delta(\mu - 1/\sqrt{3}) + \mu I_- \delta(\mu + 1/\sqrt{3})\right]\\
        &= \frac{1}{2} \left( \frac{I_+}{\sqrt{3}} - \frac{I_-}{\sqrt{3}}\right) \\
        &= \frac{I_+ - I_-}{2\sqrt{3}}
    \end{split}
\end{equation}

Combining the flux, $H$, and mean intensity, $J$, we arrive at the following relations.

\begin{equation}
    \begin{split}
        &I_+ = J + \sqrt{3}H \\
        &I_- = J - \sqrt{3}H
    \end{split}
\end{equation}

To obtain a boundary condition that can be applied to our problem, we state that the incoming flux at the surface must be equal to zero, i.e. there is no external radiation. Therefore, at the surface, $I_- = 0$ and thus $J=\sqrt{3}H$. In the next section, I will demonstrate that the derivation in \cite{2006ApJ...649...14D} does not satisfy this surface boundary condition for the flux and mean intensity, and instead satisfies the condition $J = 2H$.

\subsection{Closed-Form Solutions for Ly$\alpha$ Resonant Scattering}

My aim in this section is to understand the internal consistency of the closed-form solution for Ly$\alpha$ resonant scattering proposed by \cite{2006ApJ...649...14D}. The solution utilizes an eigenfunction expansion to separate the spatial and frequency variables; however, the spatial eigenvalues explicitly depend on the frequency. Thus, this expansion is not a true separation of variables. To assess the validity of their final result, I calculate the flux at the surface, $H$, starting from an earlier expression for the mean intensity before the approximation is made.

A few notational definitions must be made to relate various frequency and spatial variables. $x$, the distance from the source in Doppler widths, is related to the frequency variable $\sigma$ by 

\begin{equation} \label{sigma}
    \sigma = \frac{\sqrt{2\pi/27}}{a} x^3
\end{equation}

The Voigt function line profile is defined as 

\begin{equation} \label{lineprofile}
    \phi = \frac{a}{\sqrt{\pi} x^2}
\end{equation}

and

\begin{equation} \label{tau}
    R\kappa_0 = \tau_0
\end{equation}

Now, the result for the mean intensity from the end of \cite{2006ApJ...649...14D}'s derivation (Equation C17) reads

\begin{equation} \label{dijkstra}
    J(x) = \frac{\sqrt{\pi}}{\sqrt{24}a\tau_0}\left(\frac{x^2}{1 + \cosh{\sqrt{2\pi^3/27}(|x^3|/a\tau_0)}}\right)
\end{equation}

However, this has been multiplied through by $4\pi R^2$ to obtain the total emerging flux density at the surface. In our comparison, we wish to use the following expression instead.

\begin{equation} \label{c17/4piR^2}
    J(x) = \frac{1}{4\pi R^2}\frac{\sqrt{\pi}}{\sqrt{24}a\tau_0}\left(\frac{x^2}{1 + \cosh{\sqrt{2\pi^3/27}(|x^3|/a\tau_0)}}\right)
\end{equation}

This is checked against the boundary condition

\begin{equation} \label{bc}
    J = \sqrt{3} H
\end{equation}

by evaluating the surface flux derived from their Eq. C12. This equation reads 

\begin{equation} \label{c12}
    J(r, \sigma) = \frac{\sqrt{6}}{16 \pi^2 R} \frac{1}{rr_s} \sum_{n=1}^{\infty}\sin(\lambda_n r) \sin(\lambda_n r_s) \frac{\exp{(-\lambda_n |\sigma|/\kappa_0)}}{\lambda_n}
\end{equation}

Note that there is a typo in their equation, which I have resolved above. An extra factor of $\kappa_0$ appears in the numerator, which should have cancelled out in the previous step of their derivation. The quantity $\lambda_n$ is defined as 

\begin{equation} \label{lambdan}
    \lambda_n = \frac{n\pi}{R}
\end{equation}

at lowest order. Note that it is the second order term in the expansion of $\lambda_n$ that contains the line profile, which is frequency-dependent. To find the flux, we calculate the following derivative.

\begin{equation}
    \begin{split}
    H(r, \sigma) &= - \frac{1}{3\kappa_0 \phi}\frac{\partial J}{\partial r}\\
    &= - \frac{1}{3\kappa_0 \phi}\frac{\partial}{\partial r}\left(\frac{\sqrt{6}}{16 \pi^2 R} \frac{1}{rr_s} \sum_{n=1}^{\infty}\sin(\lambda_n r) \sin(\lambda_n r_s) \frac{\exp{(-\lambda_n |\sigma|/\kappa_0)}}{\lambda_n}\right)
    \end{split}
\end{equation}

Applying the product rule within the sum, we obtain

\begin{equation}
    H(r, \sigma) = - \frac{1}{3 \phi} \frac{\sqrt{6}}{16 \pi^2 R \kappa_0} \frac{1}{r_s} \sum_{n=1}^{\infty} \sin(\lambda_n r_s) \frac{\exp{(-\lambda_n |\sigma|/\kappa_0)}}{\lambda_n} \left[ \frac{-\sin(\lambda_n r)}{r^2} + \frac{\lambda_n \cos(\lambda_n r)}{r}\right]
\end{equation}

At this point, the expression is evaluated at the surface, $r=R$. Eq. \ref{lambdan} is substituted in for $\lambda_n$.

\begin{equation}
    H(\sigma) = - \frac{1}{3 \phi} \frac{\sqrt{6}}{16 \pi^2 R \kappa_0} \frac{1}{r_s} \sum_{n=1}^{\infty} \sin{\left(\frac{n\pi r_s}{R}\right)} \frac{\exp{(-n \pi |\sigma|/R\kappa_0)}}{\left(\frac{n\pi}{R}\right)} \left[ \frac{-\sin(n \pi)}{R^2} + \frac{n \pi \cos(n \pi)}{R^2}\right]
\end{equation}

The $\sin(n\pi)$ terms evaluate to zero, and the $\cos(n\pi)$ terms are equivalent to $(-1)^n$. We now have

\begin{equation}
    H(\sigma) = - \frac{1}{3 \phi} \frac{\sqrt{6}}{16 \pi^2 R^2 \kappa_0} \frac{1}{r_s} \sum_{n=1}^{\infty} (-1)^n \sin{\left(\frac{n\pi r_s}{R}\right)} \exp{(-n \pi |\sigma|/R\kappa_0)}
\end{equation}

Writing out the $\sin$ and $\exp$ terms, we are able to combine them into two separate factors raised to the power $n$, allowing the sum to be written as two geometric series.

\begin{equation}
    \sin{x} = \frac{e^{ix}-e^{-ix}}{2i}
\end{equation}

\begin{equation}
    \begin{split}
        H(\sigma) &= - \frac{1}{3 \phi} \frac{\sqrt{6}}{16 \pi^2 R^2 \kappa_0} \frac{1}{r_s} \sum_{n=1}^{\infty} (-1)^n \frac{\exp{(in\pi r_s / R)}-\exp{(-in\pi r_s/R})}{2i} \exp{(-n \pi |\sigma|/R\kappa_0)} \\
        &= - \frac{1}{3 \phi} \frac{\sqrt{6}}{32 i \pi^2 R^2 \kappa_0} \frac{1}{r_s} \sum_{n=1}^{\infty} (-1)^n \left[ \exp{\left(\frac{-\pi |\sigma|}{R\kappa_0} + \frac{i\pi r_s}{R}\right)^n}-\exp{\left(\frac{-\pi |\sigma|}{R\kappa_0} - \frac{i\pi r_s}{R}\right)^n}\right]
    \end{split}
\end{equation}

Using the geometric series

\begin{equation}
    \sum_{n=1}^{\infty}(-1)^nx^n = \frac{-x}{1+x}, \hspace{0.2cm} |x| < 1
\end{equation}

the sum can be evaluated to obtain

\begin{equation}
    H(\sigma) = - \frac{1}{3 \phi} \frac{\sqrt{6}}{32 i \pi^2 R^2 \kappa_0} \frac{1}{r_s} \left[\frac{-\exp{\left(\frac{-\pi |\sigma|}{R\kappa_0} + \frac{i\pi r_s}{R}\right)}}{1 + \exp{\left(\frac{-\pi |\sigma|}{R\kappa_0} + \frac{i\pi r_s}{R}\right)}}-\frac{-\exp{\left(\frac{-\pi |\sigma|}{R\kappa_0} - \frac{i\pi r_s}{R}\right)}}{1 + \exp{\left(\frac{-\pi |\sigma|}{R\kappa_0} - \frac{i\pi r_s}{R}\right)}}\right]
\end{equation}

Dividing top and bottom of each fraction through by its numerator, this simplifies to

\begin{equation} \label{before_trig}
    H(\sigma) = \frac{1}{3 \phi} \frac{\sqrt{6}}{32 i \pi^2 R^2 \kappa_0} \frac{1}{r_s} \left[\frac{1}{1 + \exp{\left(\frac{\pi |\sigma|}{R\kappa_0} - \frac{i\pi r_s}{R}\right)}}-\frac{1}{1 + \exp{\left(\frac{\pi |\sigma|}{R\kappa_0} + \frac{i\pi r_s}{R}\right)}}\right]
\end{equation}

Let us rewrite the terms in brackets as trigonometric functions, using the following variables as their arguments.

\begin{equation}
    \alpha = \frac{\pi |\sigma|}{R\kappa_0}; \hspace{0.4cm} \beta = \frac{\pi r_s}{R}
\end{equation}

\begin{equation}
    \begin{split}
    \frac{1}{1+e^{\alpha - i\beta}} - \frac{1}{1+e^{\alpha + i\beta}} &= \frac{1+e^{\alpha + i\beta} - 1 - e^{\alpha - i\beta}}{\left(1+e^{\alpha - i\beta}\right)\left(1+e^{\alpha + i\beta}\right)} \\
    &= \frac{e^{\alpha}\left(e^{i\beta} - e^{- i\beta}\right)}{\left(1+e^{\alpha - i\beta}\right)\left(1+e^{\alpha + i\beta}\right)} \\
    &= \frac{e^{\alpha}}{1 + e^{\alpha - i\beta} + e^{\alpha + i\beta} + e^{2\alpha}} 2i \sin{\beta} \\
    &= \frac{2i \sin{\beta}}{e^{i\beta} + e^{- i\beta} + e^{\alpha} + e^{-\alpha}} \\
    &= \frac{i \sin{\beta}}{\cos{\beta} + \cosh{\alpha}}
    \end{split}
\end{equation}

With this, Eq. \ref{before_trig} becomes

\begin{equation} \label{after_trig}
    H(\sigma) = \frac{1}{3 \phi} \frac{\sqrt{6}}{32 \pi^2 R^2 \kappa_0} \frac{1}{r_s} \left(\frac{\sin{\left(\frac{\pi r_s}{R}\right)}}{\cos{\left(\frac{\pi r_s}{R}\right)} + \cosh{\left(\frac{\pi |\sigma|}{R\kappa_0}\right)}}\right)
\end{equation}

We now take $r_s \rightarrow 0$, taking advantage of the approximation $\sin(x) \approx x$ for small $x$.

\begin{equation}
    H(\sigma) = \frac{1}{3 \phi} \frac{\sqrt{6}}{32 \pi R^3 \kappa_0} \left(\frac{1}{1 + \cosh{\left(\frac{\pi |\sigma|}{R\kappa_0}\right)}}\right)
\end{equation}

Substituting in Eq. \ref{sigma} for $\sigma$, Eq. \ref{lineprofile} for $\phi$, and Eq. \ref{tau} for $R \kappa_0$, we are left with

\begin{equation} \label{final}
    H(x) = \frac{1}{4\pi R^2}\frac{\sqrt{\pi}}{2\sqrt{24}a\tau_0} \left(\frac{x^2}{1 + \cosh{\left(\sqrt{2\pi^3/27}|x^3|/a\tau_0\right)}}\right)
\end{equation}

The two equations, Eq. \ref{final} and Eq. \ref{c17/4piR^2}, are related by the expression $J(x) = 2 H(x)$, suggesting that the two-stream boundary condition (Eq. \ref{bc}) is not satisfied.

\bibliography{bibliography}{}
\bibliographystyle{aasjournal}

\end{document}